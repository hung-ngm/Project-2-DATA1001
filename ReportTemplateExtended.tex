% Options for packages loaded elsewhere
\PassOptionsToPackage{unicode}{hyperref}
\PassOptionsToPackage{hyphens}{url}
%
\documentclass[
]{article}
\usepackage{amsmath,amssymb}
\usepackage{lmodern}
\usepackage{ifxetex,ifluatex}
\ifnum 0\ifxetex 1\fi\ifluatex 1\fi=0 % if pdftex
  \usepackage[T1]{fontenc}
  \usepackage[utf8]{inputenc}
  \usepackage{textcomp} % provide euro and other symbols
\else % if luatex or xetex
  \usepackage{unicode-math}
  \defaultfontfeatures{Scale=MatchLowercase}
  \defaultfontfeatures[\rmfamily]{Ligatures=TeX,Scale=1}
\fi
% Use upquote if available, for straight quotes in verbatim environments
\IfFileExists{upquote.sty}{\usepackage{upquote}}{}
\IfFileExists{microtype.sty}{% use microtype if available
  \usepackage[]{microtype}
  \UseMicrotypeSet[protrusion]{basicmath} % disable protrusion for tt fonts
}{}
\makeatletter
\@ifundefined{KOMAClassName}{% if non-KOMA class
  \IfFileExists{parskip.sty}{%
    \usepackage{parskip}
  }{% else
    \setlength{\parindent}{0pt}
    \setlength{\parskip}{6pt plus 2pt minus 1pt}}
}{% if KOMA class
  \KOMAoptions{parskip=half}}
\makeatother
\usepackage{xcolor}
\IfFileExists{xurl.sty}{\usepackage{xurl}}{} % add URL line breaks if available
\IfFileExists{bookmark.sty}{\usepackage{bookmark}}{\usepackage{hyperref}}
\hypersetup{
  pdftitle={Reproducible Report},
  pdfauthor={Student/s SID},
  hidelinks,
  pdfcreator={LaTeX via pandoc}}
\urlstyle{same} % disable monospaced font for URLs
\usepackage[margin=1in]{geometry}
\usepackage{color}
\usepackage{fancyvrb}
\newcommand{\VerbBar}{|}
\newcommand{\VERB}{\Verb[commandchars=\\\{\}]}
\DefineVerbatimEnvironment{Highlighting}{Verbatim}{commandchars=\\\{\}}
% Add ',fontsize=\small' for more characters per line
\usepackage{framed}
\definecolor{shadecolor}{RGB}{248,248,248}
\newenvironment{Shaded}{\begin{snugshade}}{\end{snugshade}}
\newcommand{\AlertTok}[1]{\textcolor[rgb]{0.94,0.16,0.16}{#1}}
\newcommand{\AnnotationTok}[1]{\textcolor[rgb]{0.56,0.35,0.01}{\textbf{\textit{#1}}}}
\newcommand{\AttributeTok}[1]{\textcolor[rgb]{0.77,0.63,0.00}{#1}}
\newcommand{\BaseNTok}[1]{\textcolor[rgb]{0.00,0.00,0.81}{#1}}
\newcommand{\BuiltInTok}[1]{#1}
\newcommand{\CharTok}[1]{\textcolor[rgb]{0.31,0.60,0.02}{#1}}
\newcommand{\CommentTok}[1]{\textcolor[rgb]{0.56,0.35,0.01}{\textit{#1}}}
\newcommand{\CommentVarTok}[1]{\textcolor[rgb]{0.56,0.35,0.01}{\textbf{\textit{#1}}}}
\newcommand{\ConstantTok}[1]{\textcolor[rgb]{0.00,0.00,0.00}{#1}}
\newcommand{\ControlFlowTok}[1]{\textcolor[rgb]{0.13,0.29,0.53}{\textbf{#1}}}
\newcommand{\DataTypeTok}[1]{\textcolor[rgb]{0.13,0.29,0.53}{#1}}
\newcommand{\DecValTok}[1]{\textcolor[rgb]{0.00,0.00,0.81}{#1}}
\newcommand{\DocumentationTok}[1]{\textcolor[rgb]{0.56,0.35,0.01}{\textbf{\textit{#1}}}}
\newcommand{\ErrorTok}[1]{\textcolor[rgb]{0.64,0.00,0.00}{\textbf{#1}}}
\newcommand{\ExtensionTok}[1]{#1}
\newcommand{\FloatTok}[1]{\textcolor[rgb]{0.00,0.00,0.81}{#1}}
\newcommand{\FunctionTok}[1]{\textcolor[rgb]{0.00,0.00,0.00}{#1}}
\newcommand{\ImportTok}[1]{#1}
\newcommand{\InformationTok}[1]{\textcolor[rgb]{0.56,0.35,0.01}{\textbf{\textit{#1}}}}
\newcommand{\KeywordTok}[1]{\textcolor[rgb]{0.13,0.29,0.53}{\textbf{#1}}}
\newcommand{\NormalTok}[1]{#1}
\newcommand{\OperatorTok}[1]{\textcolor[rgb]{0.81,0.36,0.00}{\textbf{#1}}}
\newcommand{\OtherTok}[1]{\textcolor[rgb]{0.56,0.35,0.01}{#1}}
\newcommand{\PreprocessorTok}[1]{\textcolor[rgb]{0.56,0.35,0.01}{\textit{#1}}}
\newcommand{\RegionMarkerTok}[1]{#1}
\newcommand{\SpecialCharTok}[1]{\textcolor[rgb]{0.00,0.00,0.00}{#1}}
\newcommand{\SpecialStringTok}[1]{\textcolor[rgb]{0.31,0.60,0.02}{#1}}
\newcommand{\StringTok}[1]{\textcolor[rgb]{0.31,0.60,0.02}{#1}}
\newcommand{\VariableTok}[1]{\textcolor[rgb]{0.00,0.00,0.00}{#1}}
\newcommand{\VerbatimStringTok}[1]{\textcolor[rgb]{0.31,0.60,0.02}{#1}}
\newcommand{\WarningTok}[1]{\textcolor[rgb]{0.56,0.35,0.01}{\textbf{\textit{#1}}}}
\usepackage{longtable,booktabs,array}
\usepackage{calc} % for calculating minipage widths
% Correct order of tables after \paragraph or \subparagraph
\usepackage{etoolbox}
\makeatletter
\patchcmd\longtable{\par}{\if@noskipsec\mbox{}\fi\par}{}{}
\makeatother
% Allow footnotes in longtable head/foot
\IfFileExists{footnotehyper.sty}{\usepackage{footnotehyper}}{\usepackage{footnote}}
\makesavenoteenv{longtable}
\usepackage{graphicx}
\makeatletter
\def\maxwidth{\ifdim\Gin@nat@width>\linewidth\linewidth\else\Gin@nat@width\fi}
\def\maxheight{\ifdim\Gin@nat@height>\textheight\textheight\else\Gin@nat@height\fi}
\makeatother
% Scale images if necessary, so that they will not overflow the page
% margins by default, and it is still possible to overwrite the defaults
% using explicit options in \includegraphics[width, height, ...]{}
\setkeys{Gin}{width=\maxwidth,height=\maxheight,keepaspectratio}
% Set default figure placement to htbp
\makeatletter
\def\fps@figure{htbp}
\makeatother
\setlength{\emergencystretch}{3em} % prevent overfull lines
\providecommand{\tightlist}{%
  \setlength{\itemsep}{0pt}\setlength{\parskip}{0pt}}
\setcounter{secnumdepth}{-\maxdimen} % remove section numbering
\ifluatex
  \usepackage{selnolig}  % disable illegal ligatures
\fi

\title{Reproducible Report}
\usepackage{etoolbox}
\makeatletter
\providecommand{\subtitle}[1]{% add subtitle to \maketitle
  \apptocmd{\@title}{\par {\large #1 \par}}{}{}
}
\makeatother
\subtitle{Project 2}
\author{Student/s SID}
\date{University of Sydney \textbar{} DATA1001 \textbar{} 2021}

\begin{document}
\maketitle

{
\setcounter{tocdepth}{3}
\tableofcontents
}
\hypertarget{executive-summary}{%
\section{Executive Summary}\label{executive-summary}}

The aim of this report is to investigate any effects of self-quarantines
due to COVID-19 on people's hobbies. We have observed that, though not
significant, people dropped few of their hobbies when their quarantine
was over. We also investigated the relation between how many hobbies
people picked up and how reliant they were to their hobbies as a coping
mechanism. However we concluded that they were not related, due to low
correlation values

\hypertarget{full-report}{%
\section{Full Report}\label{full-report}}

\hypertarget{initial-data-analysis-ida}{%
\subsection{Initial Data Analysis
(IDA)}\label{initial-data-analysis-ida}}

A primary, statistical, online survey was conducted where suitable
questions for the research were asked. Although this allows for a large
amount of information to be collected with minimal effort and cost, some
limitations followed in with the advantages. Firstly, since the survey
was voluntary, a voluntary response bias and a lack of random sampling
had occurred which could lead to distorted results. Since the primary
platform in which the survey was posted is mainly used by students from
the University of Sydney, most of the data collected were aged between
17 to 23 which does not represent the overall population. On top of
this, there was also a limited number of people that had completed the
survey (29 respondents), especially within the 2-week short span
availability of the survey. Furthermore, there may have been distorted
memories of the respondents as certain questions asked about memories
from over a year ago (start of quarantine period), questions may have
been misinterpreted, and subjective opinions may arise from questions
asking the respondents to rate on a scale of 0 to 5, which could to
statistical errors.

\begin{Shaded}
\begin{Highlighting}[]
\CommentTok{\#Load our data}
\NormalTok{survey }\OtherTok{=} \FunctionTok{read.csv}\NormalTok{(}\StringTok{"HobbiesCOVID19.csv"}\NormalTok{)}

\CommentTok{\# Remove the Timestamp column as it is not necessary for our report}
\NormalTok{survey}\SpecialCharTok{$}\NormalTok{Timestamp }\OtherTok{\textless{}{-}} \ConstantTok{NULL}

\CommentTok{\# Quick look at the structure of data}
\FunctionTok{str}\NormalTok{(survey)}
\end{Highlighting}
\end{Shaded}

\begin{verbatim}
## 'data.frame':    29 obs. of  9 variables:
##  $ age           : chr  "17-23" "17-23" "17-23" "17-23" ...
##  $ gender        : chr  "Male" "Female" "Female" "Female" ...
##  $ covidhobbyno  : int  2 1 0 6 2 3 6 0 4 3 ...
##  $ covidhobbytype: chr  "Cooking/baking, Learning Japanese" "Sports/exercise, Cooking/baking" "Literature, Music, Sports/exercise, Video games, Movies/TV shows" "Literature, Music, Sports/exercise, Arts, Video games, Movies/TV shows" ...
##  $ reliance      : int  3 4 0 4 3 2 3 5 4 2 ...
##  $ continue      : chr  "Yes" "No" "Yes" "Yes" ...
##  $ nowhobbyno    : int  2 0 4 4 1 2 3 0 3 2 ...
##  $ nowhobbytype  : chr  "Cooking/baking, Learning Japanese" "" "Literature, Sports/exercise, Video games" "Music, Sports/exercise, Video games, Movies/TV shows" ...
##  $ discontinue   : chr  "" "Not enough time, Was not suitable for me/became bored" "Not enough time" "Not enough time, Bored" ...
\end{verbatim}

\begin{Shaded}
\begin{Highlighting}[]
\CommentTok{\# Quick look at top 5 rows of data}
\FunctionTok{head}\NormalTok{(survey) }
\end{Highlighting}
\end{Shaded}

\begin{verbatim}
##     age gender covidhobbyno
## 1 17-23   Male            2
## 2 17-23 Female            1
## 3 17-23 Female            0
## 4 17-23 Female            6
## 5 17-23   Male            2
## 6 17-23 Female            3
##                                                           covidhobbytype
## 1                                      Cooking/baking, Learning Japanese
## 2                                        Sports/exercise, Cooking/baking
## 3       Literature, Music, Sports/exercise, Video games, Movies/TV shows
## 4 Literature, Music, Sports/exercise, Arts, Video games, Movies/TV shows
## 5                                       Sports/exercise, Movies/TV shows
## 6                              Music, Arts, Video games, Movies/TV shows
##   reliance continue nowhobbyno
## 1        3      Yes          2
## 2        4       No          0
## 3        0      Yes          4
## 4        4      Yes          4
## 5        3      Yes          1
## 6        2      Yes          2
##                                           nowhobbytype
## 1                    Cooking/baking, Learning Japanese
## 2                                                     
## 3             Literature, Sports/exercise, Video games
## 4 Music, Sports/exercise, Video games, Movies/TV shows
## 5                                      Movies/TV shows
## 6                               Music, Movies/TV shows
##                                             discontinue
## 1                                                      
## 2 Not enough time, Was not suitable for me/became bored
## 3                                       Not enough time
## 4                                Not enough time, Bored
## 5                                    Lack of motivation
## 6                                       Not enough time
\end{verbatim}

\begin{Shaded}
\begin{Highlighting}[]
\CommentTok{\# Our data has 29 rows and 9 columns}

\CommentTok{\# Size of the data}
\FunctionTok{dim}\NormalTok{(survey)}
\end{Highlighting}
\end{Shaded}

\begin{verbatim}
## [1] 29  9
\end{verbatim}

\begin{Shaded}
\begin{Highlighting}[]
\CommentTok{\# R\textquotesingle{}s classification of survey\textquotesingle{}s data}
\FunctionTok{class}\NormalTok{(survey)}
\end{Highlighting}
\end{Shaded}

\begin{verbatim}
## [1] "data.frame"
\end{verbatim}

\begin{Shaded}
\begin{Highlighting}[]
\DocumentationTok{\#\# R\textquotesingle{}s classification of variables}
\FunctionTok{str}\NormalTok{(mtcars)}
\end{Highlighting}
\end{Shaded}

\begin{verbatim}
## 'data.frame':    32 obs. of  11 variables:
##  $ mpg : num  21 21 22.8 21.4 18.7 18.1 14.3 24.4 22.8 19.2 ...
##  $ cyl : num  6 6 4 6 8 6 8 4 4 6 ...
##  $ disp: num  160 160 108 258 360 ...
##  $ hp  : num  110 110 93 110 175 105 245 62 95 123 ...
##  $ drat: num  3.9 3.9 3.85 3.08 3.15 2.76 3.21 3.69 3.92 3.92 ...
##  $ wt  : num  2.62 2.88 2.32 3.21 3.44 ...
##  $ qsec: num  16.5 17 18.6 19.4 17 ...
##  $ vs  : num  0 0 1 1 0 1 0 1 1 1 ...
##  $ am  : num  1 1 1 0 0 0 0 0 0 0 ...
##  $ gear: num  4 4 4 3 3 3 3 4 4 4 ...
##  $ carb: num  4 4 1 1 2 1 4 2 2 4 ...
\end{verbatim}

\begin{Shaded}
\begin{Highlighting}[]
\CommentTok{\#sapply(mtcars, class)}
\end{Highlighting}
\end{Shaded}

Summary:

\hypertarget{research-question-1}{%
\subsection{Research Question 1}\label{research-question-1}}

\hypertarget{how-did-covid-19-affected-peoples-hobbies-number-of-hobbies-during-and-after-covid-bar-graph}{%
\section{How did COVID-19 affected people's hobbies → Number of hobbies
during and after covid, bar
graph}\label{how-did-covid-19-affected-peoples-hobbies-number-of-hobbies-during-and-after-covid-bar-graph}}

\begin{Shaded}
\begin{Highlighting}[]
\CommentTok{\# Number of hobbies picked up during quarantine}
\FunctionTok{barplot}\NormalTok{(}\FunctionTok{table}\NormalTok{(survey}\SpecialCharTok{$}\NormalTok{covidhobbyno), }\AttributeTok{main=}\StringTok{"Number of hobbies picked up during quarantine"}\NormalTok{, }\AttributeTok{xlab=}\StringTok{"Number of hobbies"}\NormalTok{, }\AttributeTok{ylab=}\StringTok{"Answers"}\NormalTok{)}
\end{Highlighting}
\end{Shaded}

\includegraphics{ReportTemplateExtended_files/figure-latex/unnamed-chunk-2-1.pdf}

\begin{Shaded}
\begin{Highlighting}[]
\FunctionTok{summary}\NormalTok{(survey}\SpecialCharTok{$}\NormalTok{covidhobbyno)}
\end{Highlighting}
\end{Shaded}

\begin{verbatim}
##    Min. 1st Qu.  Median    Mean 3rd Qu.    Max. 
##   0.000   2.000   2.000   2.552   3.000   6.000
\end{verbatim}

\begin{Shaded}
\begin{Highlighting}[]
\FunctionTok{mean}\NormalTok{(survey}\SpecialCharTok{$}\NormalTok{covidhobbyno)}
\end{Highlighting}
\end{Shaded}

\begin{verbatim}
## [1] 2.551724
\end{verbatim}

\begin{Shaded}
\begin{Highlighting}[]
\CommentTok{\#Number of hobbies now}
\FunctionTok{barplot}\NormalTok{(}\FunctionTok{table}\NormalTok{(survey}\SpecialCharTok{$}\NormalTok{nowhobbyno), }\AttributeTok{main=}\StringTok{"Number of hobbies now"}\NormalTok{, }\AttributeTok{xlab=}\StringTok{"Number of hobbies"}\NormalTok{, }\AttributeTok{ylab=}\StringTok{"Answers"}\NormalTok{)}
\end{Highlighting}
\end{Shaded}

\includegraphics{ReportTemplateExtended_files/figure-latex/unnamed-chunk-2-2.pdf}

\begin{Shaded}
\begin{Highlighting}[]
\FunctionTok{summary}\NormalTok{(survey}\SpecialCharTok{$}\NormalTok{nowhobbyno)}
\end{Highlighting}
\end{Shaded}

\begin{verbatim}
##    Min. 1st Qu.  Median    Mean 3rd Qu.    Max. 
##   0.000   1.000   2.000   2.103   3.000   4.000
\end{verbatim}

\begin{Shaded}
\begin{Highlighting}[]
\FunctionTok{library}\NormalTok{(multicon)}
\end{Highlighting}
\end{Shaded}

\begin{verbatim}
## Loading required package: psych
\end{verbatim}

\begin{verbatim}
## Loading required package: abind
\end{verbatim}

\begin{verbatim}
## Loading required package: foreach
\end{verbatim}

\begin{Shaded}
\begin{Highlighting}[]
\NormalTok{mean}\OtherTok{=}\FunctionTok{c}\NormalTok{(}\FunctionTok{mean}\NormalTok{(survey}\SpecialCharTok{$}\NormalTok{covidhobbyno), }\FunctionTok{mean}\NormalTok{(survey}\SpecialCharTok{$}\NormalTok{nowhobbyno))}
\NormalTok{names}\OtherTok{=} \FunctionTok{c}\NormalTok{(}\StringTok{"COVID"}\NormalTok{, }\StringTok{"Now"}\NormalTok{)}
\NormalTok{se}\OtherTok{=} \FunctionTok{c}\NormalTok{(}\FunctionTok{popsd}\NormalTok{(survey}\SpecialCharTok{$}\NormalTok{covidhobbyno)}\SpecialCharTok{/}\FunctionTok{sqrt}\NormalTok{(}\FunctionTok{length}\NormalTok{(survey}\SpecialCharTok{$}\NormalTok{covidhobbyno)), }\FunctionTok{popsd}\NormalTok{(survey}\SpecialCharTok{$}\NormalTok{nowhobbyno)}\SpecialCharTok{/}\FunctionTok{sqrt}\NormalTok{(}\FunctionTok{length}\NormalTok{(survey}\SpecialCharTok{$}\NormalTok{nowhobbyno)))}
\NormalTok{meanhobby }\OtherTok{=} \FunctionTok{data.frame}\NormalTok{(names, mean, se)}


\FunctionTok{library}\NormalTok{(ggplot2)}
\end{Highlighting}
\end{Shaded}

\begin{verbatim}
## 
## Attaching package: 'ggplot2'
\end{verbatim}

\begin{verbatim}
## The following objects are masked from 'package:psych':
## 
##     %+%, alpha
\end{verbatim}

\begin{Shaded}
\begin{Highlighting}[]
\FunctionTok{ggplot}\NormalTok{(meanhobby, }\FunctionTok{aes}\NormalTok{(}\AttributeTok{x=}\NormalTok{names, }\AttributeTok{y=}\NormalTok{mean))}\SpecialCharTok{+}\FunctionTok{labs}\NormalTok{(}\AttributeTok{title=}\StringTok{"Comparison between during quarantine and now"}\NormalTok{)}\SpecialCharTok{+}\FunctionTok{geom\_bar}\NormalTok{(}\AttributeTok{stat=}\StringTok{\textquotesingle{}identity\textquotesingle{}}\NormalTok{)}\SpecialCharTok{+}\FunctionTok{geom\_errorbar}\NormalTok{( }\FunctionTok{aes}\NormalTok{(}\AttributeTok{x=}\NormalTok{names, }\AttributeTok{ymin=}\NormalTok{mean}\SpecialCharTok{{-}}\NormalTok{se, }\AttributeTok{ymax=}\NormalTok{mean}\SpecialCharTok{+}\NormalTok{se), }\AttributeTok{width=}\FloatTok{0.2}\NormalTok{, }\AttributeTok{colour=}\StringTok{"black"}\NormalTok{, }\AttributeTok{alpha=}\FloatTok{0.9}\NormalTok{, }\AttributeTok{size=}\FloatTok{0.8}\NormalTok{)}\SpecialCharTok{+}\FunctionTok{theme}\NormalTok{(}\AttributeTok{plot.title =} \FunctionTok{element\_text}\NormalTok{(}\AttributeTok{size=}\DecValTok{16}\NormalTok{, }\AttributeTok{face=}\StringTok{"bold.italic"}\NormalTok{,}\AttributeTok{hjust=}\FloatTok{0.5}\NormalTok{))}
\end{Highlighting}
\end{Shaded}

\includegraphics{ReportTemplateExtended_files/figure-latex/unnamed-chunk-3-1.pdf}

Summary:

\hypertarget{research-question-2}{%
\subsection{Research Question 2}\label{research-question-2}}

\hypertarget{what-kind-of-hobbies-people-start-doing-during-covid}{%
\section{What kind of hobbies people start doing during
Covid?}\label{what-kind-of-hobbies-people-start-doing-during-covid}}

\begin{Shaded}
\begin{Highlighting}[]
\NormalTok{covidhobbytypes}\OtherTok{=}\FunctionTok{strsplit}\NormalTok{(survey}\SpecialCharTok{$}\NormalTok{covidhobbytype, }\StringTok{", "}\NormalTok{)}
\NormalTok{covidhobbytypes}\OtherTok{=}\FunctionTok{table}\NormalTok{(}\FunctionTok{unlist}\NormalTok{(covidhobbytypes))}

\CommentTok{\# Install packages ggplot2}
\FunctionTok{library}\NormalTok{(ggplot2)}

\NormalTok{covidtypes }\OtherTok{=} \FunctionTok{data.frame}\NormalTok{(covidhobbytypes)}
\FunctionTok{names}\NormalTok{(covidtypes)[}\FunctionTok{names}\NormalTok{(covidtypes) }\SpecialCharTok{==} \StringTok{"Var1"}\NormalTok{] }\OtherTok{\textless{}{-}} \StringTok{"Hobbies"}
\FunctionTok{names}\NormalTok{(covidtypes)[}\FunctionTok{names}\NormalTok{(covidtypes) }\SpecialCharTok{==} \StringTok{"Freq"}\NormalTok{] }\OtherTok{\textless{}{-}} \StringTok{"Answers"}
\NormalTok{p1 }\OtherTok{=} \FunctionTok{ggplot}\NormalTok{(covidtypes, }\FunctionTok{aes}\NormalTok{(}\AttributeTok{x =}\NormalTok{ Hobbies, }\AttributeTok{y =}\NormalTok{ Answers)) }\SpecialCharTok{+} \FunctionTok{geom\_bar}\NormalTok{(}\AttributeTok{stat =} \StringTok{"identity"}\NormalTok{)}\SpecialCharTok{+} \FunctionTok{labs}\NormalTok{(}\AttributeTok{title=}\StringTok{"Kind of hobbies people start during Covid?"}\NormalTok{) }\SpecialCharTok{+} \FunctionTok{theme}\NormalTok{(}
  \AttributeTok{axis.text.x =} \FunctionTok{element\_text}\NormalTok{(}\AttributeTok{angle =} \DecValTok{90}\NormalTok{, }\AttributeTok{vjust =} \FloatTok{0.5}\NormalTok{, }\AttributeTok{hjust=}\DecValTok{1}\NormalTok{),}
  \AttributeTok{plot.title =} \FunctionTok{element\_text}\NormalTok{(}\AttributeTok{size=}\DecValTok{16}\NormalTok{, }\AttributeTok{face=}\StringTok{"bold.italic"}\NormalTok{,}\AttributeTok{hjust=}\FloatTok{0.5}\NormalTok{)                                                                )}
\NormalTok{p1}
\end{Highlighting}
\end{Shaded}

\includegraphics{ReportTemplateExtended_files/figure-latex/unnamed-chunk-4-1.pdf}

\hypertarget{what-kinds-of-hobbies-people-still-do-after-quarantine}{%
\section{What kinds of hobbies people still do after
quarantine?}\label{what-kinds-of-hobbies-people-still-do-after-quarantine}}

\begin{Shaded}
\begin{Highlighting}[]
\NormalTok{nowhobbytypes}\OtherTok{=}\FunctionTok{strsplit}\NormalTok{(survey}\SpecialCharTok{$}\NormalTok{nowhobbytype, }\StringTok{", "}\NormalTok{)}
\NormalTok{nowhobbytypes}\OtherTok{=}\FunctionTok{table}\NormalTok{(}\FunctionTok{unlist}\NormalTok{(nowhobbytypes))}

\NormalTok{nowtypes }\OtherTok{=} \FunctionTok{data.frame}\NormalTok{(nowhobbytypes)}
\FunctionTok{names}\NormalTok{(nowtypes)[}\FunctionTok{names}\NormalTok{(nowtypes) }\SpecialCharTok{==} \StringTok{"Var1"}\NormalTok{] }\OtherTok{\textless{}{-}} \StringTok{"Hobbies"}
\FunctionTok{names}\NormalTok{(nowtypes)[}\FunctionTok{names}\NormalTok{(nowtypes) }\SpecialCharTok{==} \StringTok{"Freq"}\NormalTok{] }\OtherTok{\textless{}{-}} \StringTok{"Answers"}
\NormalTok{p2 }\OtherTok{=} \FunctionTok{ggplot}\NormalTok{(nowtypes, }\FunctionTok{aes}\NormalTok{(}\AttributeTok{x =}\NormalTok{ Hobbies, }\AttributeTok{y =}\NormalTok{ Answers)) }\SpecialCharTok{+} \FunctionTok{geom\_bar}\NormalTok{(}\AttributeTok{stat =} \StringTok{"identity"}\NormalTok{)}\SpecialCharTok{+} \FunctionTok{labs}\NormalTok{(}\AttributeTok{title=}\StringTok{"Kind of hobbies people still do after quarantine"}\NormalTok{) }\SpecialCharTok{+} \FunctionTok{theme}\NormalTok{(}\AttributeTok{axis.text.x =} \FunctionTok{element\_text}\NormalTok{(}\AttributeTok{angle =} \DecValTok{90}\NormalTok{, }\AttributeTok{vjust =} \FloatTok{0.5}\NormalTok{, }\AttributeTok{hjust=}\DecValTok{1}\NormalTok{),}
      \AttributeTok{plot.title =} \FunctionTok{element\_text}\NormalTok{(}\AttributeTok{size=}\DecValTok{16}\NormalTok{, }\AttributeTok{face=}\StringTok{"bold.italic"}\NormalTok{, }\AttributeTok{hjust=}\FloatTok{0.5}\NormalTok{))}
\NormalTok{p2}
\end{Highlighting}
\end{Shaded}

\includegraphics{ReportTemplateExtended_files/figure-latex/unnamed-chunk-5-1.pdf}

\hypertarget{research-question-3}{%
\subsection{Research Question 3}\label{research-question-3}}

\hypertarget{is-there-a-linear-relationship-between-the-number-of-hobbies-and-peoples-reliance-of-hobbies-number-of-hobbies-vs-comfort-level}{%
\section{Is there a linear relationship between the number of hobbies
and people's reliance of hobbies (number of hobbies vs comfort
level)}\label{is-there-a-linear-relationship-between-the-number-of-hobbies-and-peoples-reliance-of-hobbies-number-of-hobbies-vs-comfort-level}}

\begin{Shaded}
\begin{Highlighting}[]
\CommentTok{\# Construct a scatter plot}
\FunctionTok{library}\NormalTok{(ggplot2)}
\FunctionTok{plot}\NormalTok{(survey}\SpecialCharTok{$}\NormalTok{covidhobbyno, survey}\SpecialCharTok{$}\NormalTok{reliance, }\AttributeTok{xlab =} \StringTok{"Number of hobbies"}\NormalTok{, }\AttributeTok{ylab =} \StringTok{"Reliance"}\NormalTok{)}

\CommentTok{\# Calculate the linear regression model to draw on the scatter plot}
\NormalTok{L }\OtherTok{=} \FunctionTok{lm}\NormalTok{(survey}\SpecialCharTok{$}\NormalTok{covidhobbyno }\SpecialCharTok{\textasciitilde{}}\NormalTok{ survey}\SpecialCharTok{$}\NormalTok{reliance)}

\FunctionTok{summary}\NormalTok{(L)}
\end{Highlighting}
\end{Shaded}

\begin{verbatim}
## 
## Call:
## lm(formula = survey$covidhobbyno ~ survey$reliance)
## 
## Residuals:
##     Min      1Q  Median      3Q     Max 
## -2.9785 -0.7450 -0.2779  0.7221  3.4885 
## 
## Coefficients:
##                 Estimate Std. Error t value Pr(>|t|)  
## (Intercept)       1.8109     0.7267   2.492   0.0191 *
## survey$reliance   0.2335     0.2122   1.100   0.2809  
## ---
## Signif. codes:  0 '***' 0.001 '**' 0.01 '*' 0.05 '.' 0.1 ' ' 1
## 
## Residual standard error: 1.473 on 27 degrees of freedom
## Multiple R-squared:  0.04291,    Adjusted R-squared:  0.007466 
## F-statistic: 1.211 on 1 and 27 DF,  p-value: 0.2809
\end{verbatim}

\begin{Shaded}
\begin{Highlighting}[]
\NormalTok{L}\SpecialCharTok{$}\NormalTok{coeff}
\end{Highlighting}
\end{Shaded}

\begin{verbatim}
##     (Intercept) survey$reliance 
##       1.8108883       0.2335244
\end{verbatim}

\begin{Shaded}
\begin{Highlighting}[]
\FunctionTok{abline}\NormalTok{(L)}
\end{Highlighting}
\end{Shaded}

\includegraphics{ReportTemplateExtended_files/figure-latex/unnamed-chunk-6-1.pdf}

\begin{Shaded}
\begin{Highlighting}[]
\CommentTok{\# Caluculate the linear correlation coefficient}
\FunctionTok{cor}\NormalTok{(survey}\SpecialCharTok{$}\NormalTok{covidhobbyno, survey}\SpecialCharTok{$}\NormalTok{reliance)}
\end{Highlighting}
\end{Shaded}

\begin{verbatim}
## [1] 0.2071563
\end{verbatim}

\begin{Shaded}
\begin{Highlighting}[]
\CommentTok{\#residual}
\FunctionTok{plot}\NormalTok{(survey}\SpecialCharTok{$}\NormalTok{covidhobbyno,L}\SpecialCharTok{$}\NormalTok{residuals, }\AttributeTok{xlab =} \StringTok{"Number of hobbies"}\NormalTok{, }\AttributeTok{ylab =} \StringTok{"Reliance"}\NormalTok{)}
\FunctionTok{abline}\NormalTok{(}\AttributeTok{h =} \DecValTok{0}\NormalTok{, }\AttributeTok{col =} \StringTok{"blue"}\NormalTok{)}
\end{Highlighting}
\end{Shaded}

\includegraphics{ReportTemplateExtended_files/figure-latex/unnamed-chunk-6-2.pdf}

Summary:

\hypertarget{references}{%
\section{References}\label{references}}

APA:

\begin{itemize}
\tightlist
\item
  Krause, A.E., Dimmock, J., Rebar, A.L. and Jackson, B. (2021). Music
  Listening Predicted Improved Life Satisfaction in University Students
  During Early Stages of the COVID-19 Pandemic. Frontiers in Psychology,
  11.
\item
  Michael Merschel (2020). Your pandemic hobby might be doing more good
  than you know. American Heart Association News. Accessed 20 April
  2021.
\end{itemize}

\hypertarget{beyond-the-basics---extending-your-abilities-with-rmarkdown}{%
\section{Beyond the Basics - extending your abilities with
RMarkdown}\label{beyond-the-basics---extending-your-abilities-with-rmarkdown}}

This quick reference guide will cover some basic RMarkdown for use in
your projects.

\hypertarget{lists}{%
\subsection{Lists}\label{lists}}

Here is a basic list:

\begin{itemize}
\item
  To do 1
\item
  To do 2
\item
  To do 3
\end{itemize}

\hypertarget{links}{%
\subsection{Links}\label{links}}

Here is a link to interesting web page:

\href{https://www.ted.com/topics/visualizations}{Interesting Web Page}

\hypertarget{tables}{%
\subsection{Tables}\label{tables}}

Here is a simple table.

\begin{longtable}[]{@{}lcr@{}}
\toprule
Tables & Are & Cool \\
\midrule
\endhead
col 3 is & right-aligned & \$1600 \\
col 2 is & centered & \$12 \\
zebra stripes & are neat & \$1 \\
\bottomrule
\end{longtable}

\hypertarget{images}{%
\subsection{Images}\label{images}}

Here is am image. It has not been adjusted in the rmd file, so
represents the true size of the original image. This image is sourced
directly from an online url.

To learn more about adding images directly from your own computer, see
the comments in this rmd file.

\includegraphics{https://petcube.com/blog/content/images/2017/08/kitten-supplies-cover.jpg}

Image source:
\url{https://petcube.com/blog/10-all-important-kitten-supplies-infographic/}

\hypertarget{videos}{%
\subsection{Videos}\label{videos}}

Below you will find a video embedded into your RMarkdown file. Change
the YouTube link in the rmd file to get a different video.

\hypertarget{latex}{%
\subsection{LaTeX}\label{latex}}

You can even use LaTeX in an RMarkdown document!

For example, how could you work out \(\sum_{i=1}^{5} x_{i}^3\)?

\hypertarget{r-code}{%
\subsection{R Code}\label{r-code}}

Here is an R code chunk:

Try the following commands in R.

\begin{Shaded}
\begin{Highlighting}[]
\DecValTok{1}\SpecialCharTok{+} \FunctionTok{exp}\NormalTok{(}\DecValTok{3}\NormalTok{) }\SpecialCharTok{+} \FunctionTok{sin}\NormalTok{(}\FloatTok{0.5}\NormalTok{)}
\NormalTok{x}\OtherTok{=}\FunctionTok{c}\NormalTok{(}\DecValTok{1}\NormalTok{,}\DecValTok{2}\NormalTok{,}\DecValTok{3}\NormalTok{)}
\NormalTok{x}\SpecialCharTok{\^{}}\DecValTok{2}
\FunctionTok{sum}\NormalTok{(x)}
\end{Highlighting}
\end{Shaded}

Here is some in-line code \texttt{in-line\ code}. You can put any R code
here for display, e.g.~\texttt{sum(x)}

\hypertarget{rmarkdown-resources}{%
\section{RMarkdown Resources}\label{rmarkdown-resources}}

Check out the resources below for more information on RMarkdown.

\href{https://www.oreilly.com/learning/easy-reproducible-reports-with-r?utm_medium=social\&utm_source=twitter.com\&utm_campaign=lgen\&utm_content=data+webcast+ki\&cmp=tw-data-na-article-lgen_tw_webcast_ki}{How
to use R Markdown}

\href{https://guides.github.com/features/mastering-markdown/}{Mastering
Markdown}

\end{document}
